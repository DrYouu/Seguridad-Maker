\documentclass[a4paper,12pt]{article}
\usepackage[utf8]{inputenc}
\usepackage{hyperref}
\usepackage{geometry}
\geometry{margin=2.5cm}
\usepackage{enumitem}

\title{\textbf{Charla de Seguridad Digital para el Mundo Maker\\ \large "Entender los riesgos para protegerse sin paranoia"}}
\author{[DrYouu]}
\date{}

\begin{document}

\maketitle

\tableofcontents
\newpage

\section{Introducción: Tecnología y Seguridad en el Mundo Maker}

Vivimos en una época fascinante para los makers: tenemos acceso a tecnologías que hace apenas unos años solo estaban en manos de gobiernos, laboratorios o grandes empresas. Podemos fabricar objetos, automatizar procesos, abrir puertas, crear redes de comunicación o incluso clonar dispositivos, todo con herramientas al alcance de cualquiera.

Pero esa accesibilidad también abre un nuevo frente de exposición: lo que podemos construir, otros también pueden explotarlo si no lo diseñamos con seguridad en mente.

El objetivo de esta charla es:

\begin{itemize}
    \item Mostrar de forma práctica cómo herramientas como el Flipper Zero pueden ser utilizadas contra nosotros.
    \item Aprender a protegernos sin entrar en paranoia.
    \item Concienciar sobre los riesgos físicos derivados de vulnerabilidades digitales.
    \item Abordar la ética del hackeo y el uso responsable de herramientas de pentesting.
\end{itemize}

Además, lo haremos de forma participativa, con ejemplos escenificados y algunas pequeñas sorpresas prácticas sobre los asistentes.

\section{El Flipper Zero: El Cuchillo Suizo del Pentesting Personal}

\subsection{¿Qué es el Flipper Zero?}

El Flipper Zero es un dispositivo multifunción diseñado inicialmente para investigadores de seguridad. Compacto, portable, versátil y muy intuitivo, ha ganado enorme popularidad tanto en entornos de pentesting como entre curiosos tecnológicos.

Permite interactuar con múltiples protocolos inalámbricos y electrónicos, muchos de ellos presentes en la vida cotidiana de cualquier ciudadano moderno.

\subsection{Sus funciones principales incluyen:}

\begin{itemize}
    \item \textbf{Sub-GHz}: copiar y retransmitir señales de mandos a distancia de garajes, persianas, sistemas de alarma, etc.
    \item \textbf{RFID/NFC}: clonar tarjetas de acceso de edificios, oficinas, hoteles, gimnasios.
    \item \textbf{Infrarrojos (IR)}: controlar dispositivos electrónicos mediante señales de mando.
    \item \textbf{USB HID (BadUSB)}: simular un teclado o ratón al conectarlo a un ordenador.
    \item \textbf{GPIO}: interactuar directamente con hardware externo.
    \item \textbf{Bluetooth/BLE}: interactuar con dispositivos cercanos.
\end{itemize}

\subsection{Desde el punto de vista de la víctima}

Estas capacidades, usadas sin consentimiento, permiten:

\begin{itemize}
    \item Interceptar y clonar dispositivos de acceso físico.
    \item Ejecutar comandos en dispositivos ajenos.
    \item Interferir en sistemas electrónicos sin contacto físico directo.
\end{itemize}

\newpage

\section{Escenificación 1: El Hotel Comprometido}

\subsection*{Personajes}

\begin{itemize}
    \item \textbf{Víctima}: huésped en un hotel.
    \item \textbf{Atacante}: persona aparentemente amable en el vestíbulo.
    \item \textbf{Narrador}: introduce y explica la escena (puedes ser tú misma).
\end{itemize}

\subsection*{Guion escénico}

\textbf{Narrador:}

\textit{Imaginemos una situación real que puede ocurrir en cualquier hotel. Nuestra víctima ha llegado al hotel y va a desayunar tranquilamente.}

\textbf{(La víctima se sienta en la mesa y deja su tarjeta RFID de la habitación en la mesa, junto al móvil.)}

\textbf{Narrador:}

\textit{La tarjeta de acceso RFID está ahí, expuesta. Entra ahora el atacante. No necesita tocar nada. Solo pasa a escasos centímetros con un dispositivo similar a este (muestra el Flipper Zero) en modo de lectura de tarjetas.}

\textbf{(El atacante simula escanear con disimulo mientras pasa cerca.)}

\textbf{Narrador:}

\textit{En segundos, ha copiado la tarjeta. Más tarde puede acercarse a la habitación cuando la víctima no esté y acceder sin dejar rastro de entrada forzada.}

\subsection*{Mensaje clave}

\begin{itemize}
    \item Las tarjetas RFID de baja frecuencia (125 KHz) se copian fácilmente.
    \item Nadie sospecha de alguien que simplemente pasa cerca.
    \item Existen fundas protectoras baratas que bloquean este tipo de lectura.
\end{itemize}

\newpage
