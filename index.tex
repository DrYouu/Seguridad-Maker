\documentclass[a4paper,12pt]{article}
\usepackage[utf8]{inputenc}
\usepackage{hyperref}
\usepackage{geometry}
\geometry{margin=2.5cm}
\usepackage{enumitem}

\title{\textbf{Charla de Seguridad Digital para el Mundo Maker\\ \large "Entender los riesgos para protegerse sin paranoia"}}
\author{[DrYouu]}
\date{}

\begin{document}

\maketitle

\tableofcontents
\newpage

\section{Introducción: Tecnología y Seguridad en el Mundo Maker}

En el ecosistema maker convergen creatividad, hardware abierto, protocolos diversos y, en muchas ocasiones, ausencia de controles de seguridad básicos. Esto abre puertas tanto a la innovación como a la exposición involuntaria a riesgos digitales con impacto físico.

\begin{itemize}
    \item ¿Por qué importa la seguridad digital hoy?
    \item El perfil maker: creatividad, experimentación, exposición.
    \item Riesgos inadvertidos por exceso de confianza en la tecnología.
\end{itemize}

\section{El Flipper Zero: El Cuchillo Suizo del Pentesting Personal}

\subsection{¿Qué es el Flipper Zero?}
Dispositivo multifunción de pentesting orientado inicialmente a investigadores de seguridad. Actualmente es ampliamente accesible para cualquier persona interesada.

\subsection{Funciones principales}
\begin{itemize}
    \item Sub-GHz (mandos, garajes, sensores)
    \item RFID (tarjetas de acceso)
    \item Infrared (controles remotos)
    \item GPIO hacking (hardware hacking)
    \item USB HID (BadUSB)
    \item Bluetooth/BLE
\end{itemize}

\subsection{Add-ons y expansión}
\begin{itemize}
    \item Wi-Fi Devboard
    \item Antenas externas
    \item Scripts y firmwares alternativos
\end{itemize}

\subsection{Desde el punto de vista de la víctima}
\begin{itemize}
    \item Robo de señales
    \item Clonación de accesos
    \item Spoofing y suplantación
    \item Interacciones físicas peligrosas
\end{itemize}

\section{Amenazas Digitales con Impacto en el Mundo Físico}

\begin{itemize}
    \item Cerraduras electrónicas
    \item Controles de acceso a edificios
    \item Garajes, barreras y portones
    \item Dispositivos DIY sin protocolos seguros
\end{itemize}

\section{Hackeo Social: Atacar a la Persona}

\subsection{Ingeniería social aplicada al mundo maker}
\begin{itemize}
    \item Suplantación de roles
    \item Observación de rutinas
    \item Interacción física encubierta
\end{itemize}

\subsection{Ataques combinados}
\begin{itemize}
    \item Evil Twin Wi-Fi
    \item HID scripts en eventos presenciales
    \item Escucha pasiva de frecuencias RFID
\end{itemize}

\section{Cómo Protegerse sin Paranoia}

\subsection{Prácticas físicas}
\begin{itemize}
    \item Fundas Faraday
    \item Gestión de dispositivos de acceso
    \item Minimización de señales radiadas
\end{itemize}

\subsection{Prácticas digitales}
\begin{itemize}
    \item Firmware actualizado
    \item Autenticación multifactor
    \item Desactivar interfaces no usadas
\end{itemize}

\section{Ética Hacker y Cultura del Pentesting}

\begin{itemize}
    \item Diferenciar hacker de criminal
    \item Principios de divulgación responsable
    \item Ética en el uso de herramientas de pentesting
\end{itemize}

\section{Comparativa de Dispositivos de Pentesting}

\begin{tabular}{|l|c|c|c|}
\hline
\textbf{Dispositivo} & \textbf{Precio (€)} & \textbf{Dificultad} & \textbf{Riesgo ético} \\
\hline
Flipper Zero & 150-250 & Media & Alto \\
Rubber Ducky & 50-100 & Baja & Alto \\
Wi-Fi Pineapple & 200-300 & Alta & Alto \\
LAN Turtle & 100-150 & Media & Medio \\
Bash Bunny & 120-200 & Media & Alto \\
ESP32 devkits & 5-20 & Alta & Variable \\
\hline
\end{tabular}

\section{Casos Prácticos y Escenificaciones con Actores}

\begin{itemize}
    \item Clonación de tarjeta de hotel
    \item Interferencia en apertura de garajes
    \item Red Wi-Fi trampa en evento maker
\end{itemize}

\section{Demostraciones Reales sobre el Público}

\begin{itemize}
    \item Demostración controlada de ataques con consentimiento
    \item Feedback inmediato del público
\end{itemize}

\section{Sorteo de Premios: Hackeados Premiados}

\begin{itemize}
    \item Selección aleatoria entre participantes hackeados
    \item Premios relacionados con el mundo maker y la ciberseguridad
\end{itemize}

\section{Cierre y Reflexión Final}

\begin{itemize}
    \item Educación y concienciación como mejores defensas
    \item Seguridad como parte del diseño maker
    \item Compromiso ético en el conocimiento adquirido
\end{itemize}

\section{Recursos Adicionales y Bibliografía}

\begin{itemize}
    \item \textit{Hacking: The Art of Exploitation}
    \item \textit{The Hacker Playbook}
    \item Plataformas: Hack The Box, TryHackMe, Root Me
    \item Foros: Hackaday, r/flipperzero, DEFCON, CCC
\end{itemize}

\end{document}
