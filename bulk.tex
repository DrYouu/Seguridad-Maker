\documentclass[a4paper,12pt]{article}
\usepackage[utf8]{inputenc}
\usepackage{hyperref}
\usepackage{geometry}
\geometry{margin=2.5cm}
\usepackage{enumitem}
\usepackage{tabularx}
\usepackage{longtable}

\title{\textbf{Charla de Seguridad Digital para el Mundo Maker \\ \large "Entender los riesgos para protegerse sin paranoia"}}
\author{[DrYouu]}
\date{}

\begin{document}

\maketitle

\tableofcontents
\newpage

\section{Introducción: Tecnología y Seguridad en el Mundo Maker}

Vivimos en la era dorada del acceso a la tecnología. La comunidad maker puede hoy fabricar, programar, automatizar y construir sistemas con herramientas que antes sólo poseían gobiernos o grandes corporaciones. Pero esa accesibilidad también tiene una contrapartida: el riesgo.

La seguridad digital ya no es algo exclusivo del mundo informático. Muchas amenazas digitales tienen consecuencias directas en el mundo físico: puertas que se abren, alarmas que se desactivan, dispositivos que actúan sin nuestra intervención.

El objetivo de esta charla es:

\begin{itemize}
    \item Mostrar de forma práctica cómo se explotan vulnerabilidades físicas y digitales.
    \item Enseñar cómo protegerse de forma realista, evitando la paranoia.
    \item Reflexionar sobre la ética del uso de herramientas de pentesting.
\end{itemize}

\section{El Flipper Zero: El Cuchillo Suizo del Pentesting Personal}

\subsection{¿Qué es el Flipper Zero?}

Un dispositivo multifunción portátil, diseñado para investigación de seguridad pero popularizado masivamente.

\begin{itemize}
    \item Compacto, con interfaz intuitiva.
    \item Permite interactuar con protocolos muy utilizados en el día a día.
\end{itemize}

\subsection{Funciones principales}

\begin{itemize}
    \item \textbf{Sub-GHz (300-928 MHz):} captura, análisis y retransmisión de señales de mandos de garaje, persianas, alarmas.
    \item \textbf{RFID (125 KHz y 13.56 MHz):} lectura, clonación y emulación de tarjetas de acceso.
    \item \textbf{Infrared (IR):} replicación de controles remotos.
    \item \textbf{USB HID:} simulación de teclado para ataques tipo BadUSB.
    \item \textbf{GPIO:} control de hardware externo.
    \item \textbf{Bluetooth BLE:} escaneo e interacción con dispositivos cercanos.
\end{itemize}

\subsection{Add-ons y expansión}

\begin{itemize}
    \item Wi-Fi Devboard (exploración de redes Wi-Fi)
    \item Antenas externas (amplificación Sub-GHz)
    \item Firmwares modificados (RogueMaster, Unleashed, etc.)
\end{itemize}

\section{Desde el punto de vista de la víctima}

\begin{itemize}
    \item \textbf{Acceso físico no autorizado:} clonación de tarjetas de hotel, oficinas, gimnasios.
    \item \textbf{Interferencia remota:} apertura de garajes, desactivación de alarmas.
    \item \textbf{Ataques BadUSB:} ejecución de comandos automatizados en segundos.
    \item \textbf{Explotación BLE:} análisis de dispositivos cercanos.
\end{itemize}

\section{Amenazas Digitales con Impacto en el Mundo Físico}

\begin{itemize}
    \item Cerraduras electrónicas vulnerables.
    \item Portones y barreras sin rolling-code.
    \item Sistemas DIY caseros sin protección básica.
    \item Dispositivos IOT sin cifrado.
\end{itemize}

\section{Hackeo Social: Atacar a la Persona}

\subsection{Ingeniería social aplicada al mundo maker}

\begin{itemize}
    \item Ganar confianza mediante conocimiento técnico aparente.
    \item Observar rutinas y hábitos.
    \item Manipular al usuario para obtener claves o acceso físico.
\end{itemize}

\subsection{Ataques combinados}

\begin{itemize}
    \item Evil Twin Wi-Fi en eventos makers.
    \item USB drop attacks (dejar dispositivos USB preparados para ser recogidos).
    \item Escucha pasiva de frecuencias abiertas.
\end{itemize}

\section{Cómo Protegerse sin Paranoia}

\subsection{Prácticas físicas}

\begin{itemize}
    \item Fundas Faraday para tarjetas.
    \item Bloqueadores físicos de señales.
    \item Desactivación de llaves digitales cuando no se usan.
\end{itemize}

\subsection{Prácticas digitales}

\begin{itemize}
    \item Actualización constante de firmware.
    \item Contraseñas robustas y autenticación múltiple.
    \item No conectar dispositivos USB de origen dudoso.
\end{itemize}

\section{Ética Hacker y Cultura del Pentesting}

\begin{itemize}
    \item Un hacker no es un criminal.
    \item Divulgación responsable: reportar vulnerabilidades.
    \item No explotar debilidades fuera de entornos controlados o con víctimas no consentidas.
\end{itemize}

\section{Comparativa de Dispositivos de Pentesting}

\begin{longtable}{|p{4cm}|p{2cm}|p{2cm}|p{3cm}|}
\hline
\textbf{Dispositivo} & \textbf{Precio (€)} & \textbf{Dificultad} & \textbf{Riesgo ético} \\
\hline
Flipper Zero & 150-250 & Media & Alto \\
Rubber Ducky & 50-100 & Baja & Alto \\
Wi-Fi Pineapple & 200-300 & Alta & Alto \\
LAN Turtle & 100-150 & Media & Medio \\
Bash Bunny & 120-200 & Media & Alto \\
ESP32 devkits & 5-20 & Alta & Variable \\
\hline
\end{longtable}

\section{Casos Prácticos y Escenificaciones con Actores}

\subsection*{Escenificación 1: El Hotel Comprometido}

\textbf{Personajes:} Víctima, Atacante, Narrador.

\textbf{Narrador:} 
\textit{La víctima desayuna tranquilamente en el hotel. La tarjeta de su habitación está expuesta en la mesa.}

\textbf{Atacante (pasa disimuladamente cerca):}
\textit{(Simula escanear la tarjeta con el Flipper Zero)}

\textbf{Narrador:} 
\textit{La tarjeta ha sido clonada en segundos.}

\textbf{Mensaje clave:} Cuidado con tarjetas RFID LF expuestas; existen fundas protectoras económicas.

\subsection*{Escenificación 2: Apertura de Garaje}

\textbf{Personajes:} Propietario, Atacante, Narrador.

\textbf{Propietario:} 
\textit{(Abre el garaje con su mando habitual.)}

\textbf{Atacante (a distancia, graba la señal):}
\textit{(Captura la señal Sub-GHz.)}

\textbf{Narrador:} 
\textit{El atacante ha grabado la señal fija y ahora puede abrir el garaje sin necesidad de fuerza.}

\textbf{Mensaje clave:} Si el mando no usa rolling-code, es trivial de clonar.

\subsection*{Escenificación 3: El USB Malicioso}

\textbf{Personajes:} Víctima, Atacante, Narrador.

\textbf{Narrador:} 
\textit{Un pendrive encontrado en un taller maker. Curioso, alguien lo conecta a su portátil.}

\textbf{Víctima:} 
\textit{(Conecta el pendrive.)}

\textbf{Atacante (por BadUSB):} 
\textit{(El Flipper Zero simula teclado y ejecuta comandos automáticamente.)}

\textbf{Mensaje clave:} No conectar dispositivos USB desconocidos.

\subsection*{Escenificación 4: Evil Twin Wi-Fi}

\textbf{Personajes:} Público del evento.

\textbf{Narrador:} 
\textit{Durante el evento hay un Wi-Fi gratuito disponible: “MakerFest\_Free”.}

\textbf{Narrador:} 
\textit{El atacante monta una copia exacta de la red con el mismo nombre y señal más potente.}

\textbf{Público (algunos se conectan a la red falsa):} 

\textbf{Narrador:} 
\textit{El atacante intercepta tráfico y credenciales de quienes cayeron en la trampa.}

\textbf{Mensaje clave:} Usar VPN en redes abiertas; comprobar autenticidad de puntos Wi-Fi.

\subsection*{Escenificación 5: Hackeo Social en evento}

\textbf{Narrador:} 
\textit{Un asistente simpático empieza a conversar con otros participantes. Pregunta sobre los dispositivos que traen, sus credenciales, cómo han configurado sus sistemas…}

\textbf{Atacante:} 
\textit{(Obtiene datos personales y técnicos valiosos sin necesidad de tecnología, sólo por conversación.)}

\textbf{Mensaje clave:} La ingeniería social es la herramienta más potente. Cuidado con lo que compartimos por simple simpatía.

\section{Demostraciones Reales sobre el Público}

\begin{itemize}
    \item Se realizarán demostraciones controladas sobre asistentes previamente avisados.
    \item Se explicarán los pasos exactos del ataque para que todos puedan aprender.
    \item El público podrá comprobar en directo los riesgos expuestos.
\end{itemize}

\section{Sorteo de Premios: Hackeados Premiados}

\begin{itemize}
    \item Aquellos que hayan sido "hackeados" durante las pruebas prácticas participarán en un sorteo de premios.
    \item Premios: fundas Faraday, protectores RFID, dispositivos de aprendizaje maker.
\end{itemize}

\section{Cierre y Reflexión Final}

\begin{itemize}
    \item Todo conocimiento técnico implica responsabilidad.
    \item Conocer las vulnerabilidades permite protegerse mejor.
    \item La seguridad es un hábito, no un estado.
    \item La paranoia es tan inútil como la ignorancia: equilibrio y criterio.
\end{itemize}

\section{Recursos Adicionales y Bibliografía}

\begin{itemize}
    \item \textit{Hacking: The Art of Exploitation} - Jon Erickson
    \item \textit{The Hacker Playbook} - Peter Kim
    \item Plataformas: Hack The Box, TryHackMe, Root Me
    \item Foros: Hackaday, Reddit r/flipperzero, DEFCON, CCC
\end{itemize}

\end{document}
